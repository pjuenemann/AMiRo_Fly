\section{Modellierung mit MatLab/Simulink}
Dieses Kapitel beschreibt die Modellierung des in \ref{kap3} präsentierten Systems mithilfe der Entwicklungssoftware Simulink von The Mathworks. Dabei wird insbesondere auf die Verwendung eines bereits existierenden Extended Kalman-Filters aus der Simulink Bibliothek eingegangen. Zudem wird erläutert, wie MatLab zur Konfiguration der Systemparameter dient.\\\\
Das gesamte System ist als zeit-diskretes Modell aufgebaut, das mit $\delta t$ in s parametrisiert werden kann. Dieser Parameter wird ebenso wie übrigen Konstanten über die \textit{text}{InitFcn} in den \textit{Callbacks} der \textit{Block Properties} definiert. Weitere Konstanten sind beispielsweise die Kovarianzmatrizen des Prozess- und Messrauschens sowie die Beobachtungsmatrix $H$ des Messdaten.\\\\

\subsection{Extended Kalman-Filter in Simulink}
Die \textit{Control System Toolbox} aus Simulink Bibliothek stellt einen ausgereiften Block zur Zustandsvorhersage für nichtlineare Systeme mithilfe des Extended Kalman-Filters bereits zur Verfügung. Dieser Filterblock zeichnet sich durch ein großes Konfigurationsspektrum aus. So können bis zu fünf Messgrößeneingänge eingestellt werden, die separat über eine Signalleitung aktiviert werden können, sobald Messdaten zur Verfügung stehen und in den Korrektor-Schritt mit eingehen sollen. Außerdem kann für jeden Eingang eine eigene Messfunktion $h(\textbf{x}_k)$ und die Kovarianz des Messrauschens definiert werden.\\\\
Neben der genannten Messfunktion kann auch die in \ref{zustandsschätzung} beschriebene Zustandsübergangsfunktion $f(\textbf{x}_{k-1}, \textbf{u}_{k-1})$ samt der Kovarianz des Prozessrauschens an den Filter übergeben werden. Hierbei kann konfiguriert werden, ob es sich um ein zeit-variantes beziehungsweise -invariantes Prozessrauschen handelt. In diesem Modell ist dieses als additives, zeit-invariantes Rauschen gewählt und mit einer Matrix $Q^{12x12}$, deren Elemente alle den Wert $0,5$ haben, schrieben.\\\\
Des Weiteren kann zur Initialisierung des Filters der Systemzustand wie auch die Kovarianzvorhersage vorgegeben werden. !!! Womit initialisiert? !!! Diese kann bei Bedarf als Ausgang des Filters konfiguriert werden. In diesem Modell findet sie allerdings keine Verwendung. Standardmäßig ist hingegen aber der vorhergesagte Zustand ein Ausgang des Systems. Entsprechend liefert dieser den Zustandsvektor $\textbf{x}_k$.

\subsection{title}