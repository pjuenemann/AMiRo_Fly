\section{Grundlagen}\label{kapitel1}
Dieses Kapitel enthält die Beschreibungen der in diesem Projekt verwendeten Systeme. Darunter fallen der AMiRo, der die Hardware zur Berechnung der Zustandsschätzung sowie der Positions- und Lageregelung bereitstellt. Ebenso gehören zu den Systemen die Hardware der Drohne mitsamt des Flight Controllers, der geregelt werden soll sowie die Telewerkbank, innerhalb der die Position der Drohne erfasst wird.
\subsection{Flight Controller}
Als zentrale Steuerungshardware wurde für die AMiRoFly-Drohne der Seriously Pro Racing F3 EVO (SPRacingF3EVO) ausgewählt. Dieser Flight Controller stammt aus dem mittleren Kostensegment und besitzt eine mit \unit[72]{MHz} getaktete CPU\footnote{Central Processing Unit} vom Typ STM32F303. Daneben sind darauf sämtliche Komponenten verbaut, die dem aktuellen Standard entsprechen. Darunter fallen ein 9-Achsen erfassender Bewegungssensor vom Typ InvenSense MPU-9250. Dieser besteht aus einer IMU\footnote{Inertial Measurement Unit} mit einem 3-Achsen Beschleunigungssensor (Accelerometer) und einem 3-Achsen Drehratensensor (Gyroskop). Ersterer besitzt in dem verwendeten Modus einen Wertebereich von $\pm \unit[16]{g}$ und ist mit einem Faktor von \unit[2048]{LSB/g} skaliert. Das Gyroskop ist mit einem Wertebereich von $\pm \unit[2000]{\frac{^\circ}{s}}$ und einem Skalierungsfaktor von $\unit[16,4]{\frac{LSB}{\frac{^\circ}{s}}}$ spezifiziert. Außerdem ist auf der MPU-9250 ein 3-Achsen Digitalkompass (Magnetometer) verbaut, dessen Wertebereich von $\pm \unit[4800]{\mu T}$ reicht und mit $\unit[0,6]{\frac{\mu T}{LSB}}$ skaliert ist. Sämtliche Skalierungsfaktoren werden in dem später beschriebenen Modell berücksichtigt. Die Kommunikation mit der CPU wird über eine SPI\footnote{Serial Peripharal Interface}-Schnittstelle vollzogen. Weiterhin besitzt der SPRacingF3EVO ein Barometer vom Typ BMP280 sowie sämtliche weitere Schnittstellen.

\subsection{AMiRo}
Der AMiRo ist ein Robotersystem, das im Rahmen von Forschungsprojekten innerhalb der Arbeitsgruppe Kognitronik \& und Sensorik entwickelt wurde. Dabei handelt es sich um einen zylinderförmigen Miniroboter, dessen Komponenten sich bedingt durch seine modulare Bauweise verschieden variieren lassen. Dadurch setzt sich der AMiRo aus verschiedenen Platinen zusammen, die jeweils einen bestimmten Zweck erfüllen.\\\\
Für dieses Projekt soll der AMiRo als Steuerungshardware für eine Drohne dienen. Dafür wurde ein AMiRo bestehend aus einem \textit{PowerBoard}, einem \textit{CognitionBoard} und einem \textit{LightRingBoard}. Das \textit{PowerBoard} ist dabei für die Stromversorgung und das gesamte elektronische Verhalten des Systems zuständig (siehe AMiRo-Paper). Dafür ist es mit einem STM32F405 Mikrocontroller ausgestattet, der einen ARM Cortex-M4 als CPU besitzt. Das \textit{CognitionBoard} hingegen besitzt einen ARM Cortex-A8, der innerhalb des Projekts, die Zustandsschätzung und Positions- sowie Lageregelung des Systems durchführt. Dafür arbeitet auf dem AMiRo ein Linux-System. Abschließend besitzt die verwendete Variante des AMiRo ein \textit{LightRingBoard}, dessen LEDs\footnote{Light-Emitting Diode} zu Debugging-Zwecken verwendet werden können. Die Kommunikation mit peripheren Systemen wird hier über eine serielle sowie eine Wi-Fi-Schnittstelle betrieben.

\subsection{Telewerkbank}
Die Telewerkbank(TWB) ist ein für Kamera-Tracking ausgelegtes Labor der Arbeitsgruppe Kognitronik \& Sensorik. Mit Hilfe von vier an der Decke platzierten Kameras, die vertikal zum Boden gerichtet sind, kann eine Fläche mit einer Größe von etwa \unit[6]{m} mal \unit[6]{m} erfasst werden. Für die TWB spezifizierte Marker können so auf dieser Fläche erkannt und ihre Position getrackt werden. Die Kommunikation innerhalb der TWB geschieht dabei über RSB\footnote{Robotics Service Bus}-Ressourcen. Um die TWB auch für das Tracking von Drohnen auszulegen, wurde das Kamerasystem modifiziert, sodass es nun möglich ist, Marker in einer Höhe von bis zu \unit[1,5]{m} zu erfassen. Das Tracking ist somit auch für den dreidimensionalen Raum ausgelegt.

\subsection{Kommunikation der Systeme}
Damit die in diesem Projekt erarbeitete Zustandsschätzung sowie Positions- und Lageregelung der Drohne durchgeführt werden kann, muss die Kommunikation der drei zuvor beschriebenen Systeme untereinander gewährleistet sein. Der AMiRo fungiert dabei als Mittelpunkt des Systems, da auf ihm die Zustandsschätzung und anschließende Regelung berechnet werden.\\\\
Wie in Kapitel Einleitung beschrieben, stellt die TWB dafür die getrackte Position der Drohne zur Verfügung. Dies geschieht über ein Wi-Fi-Netzwerk, über das der AMiRo mit der Telewerkbank kommuniziert. Über das RSB können die Daten anschließend auf dem AMiRo zur Berechnung verwendet werden.\\\\
Die bidirektionale Kommunikation zwischen dem AMiRo und dem SPRacingF3EVO, der als Hardwarebasis der Drohne fungiert, wird über eine serielle Schnittstelle hergestellt. Der AMiRo liest dabei die Sensordaten, die vom Flight Controller ausgegeben werden. In diesem Fall sind das Beschleunigungswerte, Drehraten, Magnetfeld- und Luftdruckwerte. Entgegengesetzt schreibt der AMiRo Steueranweisungen an den SPRacingF3EVO.\\\\
Das verwendete Protokoll wurde in einem verwandten Projekt entwickelt und basiert auf dem Base85-Kodierverfahren. Für einen Lesezugriff auf den Flight Controller Binärdaten dekodiert und für einen Schreibzugriff entsprechend kodiert.