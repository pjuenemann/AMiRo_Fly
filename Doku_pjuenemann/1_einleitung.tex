\section{Einleitung}\label{einleitung}
Der Bedarf nach autonom agierenden Systemen steigt kontinuierlich an. Dadurch werden Forschung und Industrie angetrieben innovative, ressourceneffiziente Systeme zu entwickeln, die sich durch autonomes Handeln in der Umwelt auszeichnen.\\\\
Der AMiRo ist solch ein System, das in der Arbeitsgruppe Kognitronik \& Sensorik der Universität Bielefeld entwickelt wird. Dabei handelt es sich um einen zylinderförmigen Miniroboter mit zwei Rädern. Mithilfe von acht Infrarot-Sensoren und anderen Modulen, wie Kamera oder Laserscanner, die wahlweise angebracht werden können, kann er sich autonom bewegen und orientieren\cite{AMiRo}.\\\\
In dem Simulationsprogramm Gazebo gibt es ebenfalls eine Nachbildung des AMiRo. Damit können jegliche Applikationen in einer realitätsnahen Umgebung auf dem AMiRo simuliert werden. Da bislang noch kein Modell mit Infrarot-Sensoren in Gazebo vorhanden ist, soll innerhalb dieses Projekts solch ein AMiRo entwickelt werden. Als Grundlage dafür dient ein Sensor vom Typ VCNL4020, der bereits im herkömmlichen Roboter verbaut ist. Das Modell in Gazebo soll ebenso mit acht dieser Sensoren erweitert werden.\\\\
Diese Dokumentation beschreibt das Vorgehen bei der Entwicklung des Simulationsmodells in vier Schritten. Zuerst wird die Aufnahme der originalen Sensordaten und deren Auswertung beschrieben. Anschließend wird der Entwurf und die Implementierung eines Näherungssensors in Gazebo sowie die des dazugehörigen Plugins erläutert. In Kapitel 4 wird das Sensormodell, das im Hintergrund realitätsnahe Sensorwerte im \unit[16]{Bit}-Format erzeugt, entwickelt und beschrieben. Die Evaluation des Modells wird in Kapitel 5 mithilfe eines Mapping-Programms durchgeführt und ausgewertet. Abschließend werden die erzielten Ergebnisse zusammengefasst.