% das Papierformat zuerst
%\documentclass[a4paper, 11pt]{article}

% deutsche Silbentrennung
%\usepackage[ngerman]{babel}

% wegen deutschen Umlauten
%\usepackage[ansinew]{inputenc}

% hier beginnt das Dokument
%\begin{document}


\thispagestyle{empty}

%\begin{figure}[t]
% \includegraphics[width=0.6\textwidth]{abb/fh_koeln_logo}
%\end{figure}


\begin{verbatim}


\end{verbatim}

\begin{center}
\Large{Universität Bielefeld}\\
\Large{Arbeitsgruppe Kognitronik \& Sensorik}\\
\end{center}


\begin{center}
\Large{Studiengang BioMechatronik}
\end{center}
\begin{verbatim}




\end{verbatim}
\begin{center}
\doublespacing
\textbf{\LARGE{\titleDocument}}\\
\singlespacing
\begin{verbatim}

\end{verbatim}
\textbf{{~\subjectDocument}}
\end{center}
\begin{verbatim}

\end{verbatim}
\begin{center}

\end{center}
\begin{verbatim}

\end{verbatim}
\begin{center}
%\textbf{zur Erlangung des akademischen Grades \\ Bachelor / Master of Science}
\end{center}
\begin{verbatim}






\end{verbatim}
\begin{flushleft}
\begin{tabular}{llll}
\textbf{Thema:} & & Zustandsvorhersage und Postionskontrolle einer Quadrotordrohne & \\
& & auf Hardwarebasis des AMiRo & \\
& & \\
\textbf{Autor:} & & Philipp Jünemann& \\
& & pjuenemann@techfak.uni-bielefeld.de & \\
& & \\
\textbf{Version vom:} & & \today &\\
& & \\
\textbf{Betreuer:} & & Timo Korthals, M.Sc. &\\
\end{tabular}
\end{flushleft}